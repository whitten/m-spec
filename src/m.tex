\documentclass{article}
\usepackage{amsmath,amsfonts,listings,syntax,fullpage,multicol,float}
\renewcommand{\ttdefault}{pcr}

\title{The M Specification}
\author{Aedan Smith}

\setlength{\parindent}{0in}

\lstset{
morekeywords={def,macro,fn,impure},
morecomment=[l]{;},
morestring=[b]",
basicstyle=\ttfamily\small,
keywordstyle=\textbf,
commentstyle=\textit,
stringstyle=\textit,
showstringspaces=false,
escapeinside={<<}{>>}
}

\floatstyle{boxed}
\restylefloat{figure}
\setlength\intextsep{0pt}

\begin{document}
    \maketitle
    \newpage

    \tableofcontents
    \newpage

    \section{Syntax}\label{sec:syntax}

    \begin{figure}[h]
        \setlength{\grammarindent}{8em}
        \centering
        \begin{grammar}
            <character> ::= $\mathbb{C}\supseteq<special + newline + whitespace>$

            <special> ::= \{`;',`"',`(',`)'\}

            <comment> ::= `;' <character - newline>* <newline>

            <symbol> ::= <character - special - whitespace>*
            \alt `"' (<character - `"'> | `\\' <character>)* `"'

            <expression> ::= <symbol>
            \alt `(' <expression>* `)'

            <program> ::= <expression>*
        \end{grammar}
        \caption{The M grammar in EBNF.}
    \end{figure}

    \begin{multicols}{2}
        \subsection{Characters}\label{subsec:characters}

        \begin{minipage}{\columnwidth}
            M can use any character set which encodes the four special characters, a backslash, a newline character, and a whitespace character.
            For consistency, all M code samples in this specification use the ASCII character set.
        \end{minipage}

        \subsection{Specials}\label{subsec:specials}

        \begin{minipage}{\columnwidth}
            The four special characters are reserved for use in other productions.
            This is done to ensure that symbols such as \lstinline$this()$ do not include any special characters which are meant be used for other syntactic constructs.
            Note that the backslash is not included in the list of special characters as it is only used inside of a literal symbol.
        \end{minipage}

        \subsection{Comments}\label{subsec:comments}

        \begin{minipage}{\columnwidth}
            Comments in M begin with a semicolon and last until the end of the line.
            They are ignored and discarded like whitespace and newlines, as they are only intended to be used for explaining code.
        \end{minipage}

        \subsection{Symbols}\label{subsec:symbols}

        \begin{minipage}{\columnwidth}
            Symbols in M have two forms, inline and literal.
            Inline symbols are strings terminated by whitespace or a special symbol, and should be used by default.
            Literal symbols are escapable strings surrounded by quotes, and should only be used when a symbol is impossible to represent using inline symbols (for example, the symbol \lstinline$"()"$).
        \end{minipage}

        \subsection{Expressions}\label{subsec:expressions}

        \begin{minipage}{\columnwidth}
            Expressions in M have two forms, symbols and lists.
            Symbol expressions are identical to symbols defined in section~\ref{subsec:symbols}.
            Lists expressions are simply lists of expressions surrounded by matching parentheses.
        \end{minipage}

        \subsection{Programs}\label{subsec:programs}

        \begin{minipage}{\columnwidth}
            M programs are lists of expressions.
            It is unspecified how these expressions are stored, so they can be in memory, in a single file, in multiple files, on the internet, or any other method which is convenient.
        \end{minipage}
    \end{multicols}
    \newpage

    \section{Semantics}\label{sec:semantics}

    \begin{figure}[h]
        \centering
        \begin{align*}
            \text{(Symbol)}&\dfrac{}{\langle x,\Gamma\rangle\Downarrow\Gamma(x)}
            &\dfrac{\langle e,\Gamma\rangle\Downarrow v}{\langle\lstinline$(macro x e)$,\Gamma\rangle\Downarrow\langle v,\Gamma(x) = v\in\mathbb{M}\rangle}&\text{(Macro)}\\
            \text{(Function)}&\dfrac{}{\lstinline$(fn x e)$\Downarrow\lambda x.e}
            &\dfrac{\langle f,\Gamma\rangle\Downarrow\lambda x.e\quad\langle a,\Gamma\rangle\Downarrow i\quad\langle e[x/i]\rangle\Downarrow v}{\langle\lstinline$(f a)$,\Gamma\rangle\Downarrow v}&\text{(Apply-Function)}\\
            \text{(Define)}&\dfrac{\langle e,\Gamma\rangle\Downarrow v}{\langle\lstinline$(def x e)$,\Gamma\rangle\Downarrow\langle v,\Gamma(x) = v\notin\mathbb{M}\rangle}
            &\dfrac{\Gamma(m)\in\mathbb{M}\quad\langle m,\Gamma\rangle\Downarrow f\quad\langle f(e),\Gamma\rangle\Downarrow v}{\langle\lstinline$(m e)$,\Gamma\rangle\Downarrow v}&\text{(Apply-Macro)}
        \end{align*}
        \caption{The natural semantics of M.}
    \end{figure}

    \begin{multicols}{2}
        \subsection{Symbol}\label{subsec:symbol}

        \begin{minipage}{\columnwidth}
            Symbol expressions are expressions of the form \lstinline$name$.
            They always evaluate to the \lstinline$value$ of the def expression that defines \lstinline$name$.
            If \lstinline$name$ is not defined, they evaluate to $\bot$.
        \end{minipage}

        \subsection{Function}\label{subsec:function}

        \begin{minipage}{\columnwidth}
            Function expressions are expressions of the form \lstinline$(fn name value)$.
            They evaluate to a function $\lambda\lstinline{name}.\lstinline{value}$.
            When applied, they perform the substitution $\lstinline{value}\left[\lstinline{name}/\lstinline{argument}\right]$.
        \end{minipage}

        \subsection{Define}\label{subsec:def}

        \begin{minipage}{\columnwidth}
            Def expressions are expressions of the form \lstinline$(def name value)$.
            They state that all symbols \lstinline$name$ evaluate to \lstinline$value$, and evaluate to the the \lstinline$value$ they define.
            Multiple def expressions with the same \lstinline$name$ are invalid.
        \end{minipage}

        \subsection{Macro}\label{subsec:macro}

        \begin{minipage}{\columnwidth}
            Macro expressions are expressions of the form \lstinline$(macro name value)$.
            They state that all symbols \lstinline$name$ evaluate to \lstinline$value$, and evaluate to the value of \lstinline$name$.
            When applied, macros evaluate \lstinline$name$ and apply it to the expression encoding of \lstinline$argument$, then evaluate the result.
        \end{minipage}

        \subsection{Apply}\label{subsec:apply}

        \begin{minipage}{\columnwidth}
            Apply expressions are expressions of the form \lstinline$(value argument)$.
            If \lstinline$value$ is a function, it performs application as described in section~\ref{subsec:function}.
            If \lstinline$value$ is a macro, it performs application as described in section~\ref{subsec:macro}
        \end{minipage}
    \end{multicols}
    \newpage

    \section{Encodings}\label{sec:encodings}

    \begin{multicols*}{2}
        \lstinputlisting{encodings.m}
    \end{multicols*}
    \newpage

    \section{Reference Implementation}\label{sec:implementation}

    \subsection{Parser}\label{subsec:parser}

    \lstinputlisting{parser.m}

    \subsection{Interpreter}\label{subsec:interpreter}

    \lstinputlisting{interpreter.m}

    \subsection{Compiler}\label{subsec:compiler}

    \lstinputlisting{compiler.m}
\end{document}