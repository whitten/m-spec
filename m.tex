\documentclass[11pt]{article}
\usepackage{amsmath,amsfonts,listings,syntax,fullpage,multicol,float}
\renewcommand{\ttdefault}{pcr}

\title{The M Specification}
\author{Aedan Smith}

\lstset{
morekeywords={def,macro,fn},
morecomment=[l]{;},
morestring=[b]",
basicstyle=\ttfamily,
keywordstyle=\textbf,
commentstyle=\textit,
stringstyle=\textit,
showstringspaces=false
}

\restylefloat{figure}
\setlength\intextsep{0pt}

\begin{document}
    \maketitle
    \newpage

    \tableofcontents
    \newpage

    \section{Introduction}\label{sec:introduction}

    There has always been a clear separation between practical programming languages and mathematical programming constructs.
    Even languages such as ML and Scheme, though rooted in mathematics, have unnecessary or impure extensions for convenience and performance.
    While these extensions improve the language in some ways, they also make it more complex;
    programs become difficult to reason about for humans and computers alike, and maintaining a compliant, feature-rich software stack becomes impossible.

    Simplicity is necessary to reason about a programming language;
    and when writing simple programs, most programming languages are.
    It is when programs get more complex, be it for performance or for abstraction, that simplicity is discarded;
    complex extensions are used, edge cases are abused, and reasoning about a program becomes impossible.

    M is not just a simple programming language --- mathematical systems like Turing Machines and the $\lambda$-Calculus are much simpler than M is.
    M is a simple programming language which can express the extensions provided by practical programming languages without the additional complexity.

    To do this, M exhibits the following properties:
    \begin{enumerate}
        \item All functions are pure.
        \item All definitions are unordered.
        \item Any two expressions can be made equivalent with a macro.
    \end{enumerate}
    \newpage

    \section{Syntax}\label{sec:syntax}

    \begin{figure*}[h]
        \centering
        \begin{align*}
            &\forall\mathbb{C}  &:&\;\mathbb{C}\supseteq\texttt{(special + newline + whitespace)}\land\texttt{null}\notin\mathbb{C}\\
            &\texttt{character} &=&\;\mathbb{C}\\
            &\texttt{special}   &=&\;\{\texttt{`;'},\texttt{`"'},\texttt{`('},\texttt{`)'}\}\\
            &\texttt{comment}   &=&\;\texttt{`;'}\quad\texttt{(character - newline)*}\quad\texttt{newline}\\
            &\texttt{symbol}    &=&\;\texttt{(character - special - whitespace)*}\\
            & &|&\;\texttt{`"'}\quad\texttt{character*}\quad\texttt{`"'}\\
            & &|&\;\texttt{`""'}\quad\texttt{character*}\quad\texttt{`""'}\\
            &\texttt{expression}&=&\;\texttt{symbol}\\
            & &|&\;\texttt{`('}\quad\texttt{expression*}\quad\texttt{`)'}\\
            &\texttt{program}   &=&\;\texttt{expression*}
        \end{align*}
        \caption{The M grammar in EBNF.}
    \end{figure*}

    \subsection{Characters}\label{subsec:characters}

    \begin{minipage}{\columnwidth}
        M can use any character set which encodes the four special characters, a newline character, and a whitespace character.
        To allow for simple source termination, the null character is not allowed in M code.
        For consistency, all M code samples in this specification use the ASCII character set.
    \end{minipage}

    \subsection{Specials}\label{subsec:specials}

    \begin{minipage}{\columnwidth}
        The four special characters are reserved for use in non-symbol syntax.
        This is done to ensure that symbols do not include any characters which are meant be used in other syntactic constructs.
        For example, if the special characters were not reserved, the expression \texttt{(symbol)} would be parsed as \texttt{("symbol)"} rather than \texttt{("symbol")}.
    \end{minipage}

    \subsection{Comments}\label{subsec:comments}

    \begin{minipage}{\columnwidth}
        Comments in M begin with a semicolon and last until the end of the line.
        They are ignored and discarded like whitespace and newlines, as they are only intended to be used for explaining code.
        M does not provide multiline comments, as comments are not meant to be used for documentation.
    \end{minipage}

    \subsection{Symbols}\label{subsec:symbols}

    \begin{minipage}{\columnwidth}
        Symbols in M have two forms, inline and literal.
        Inline symbols are strings terminated by whitespace or a special symbol, and should be used by default.
        Literal symbols are strings surrounded by one or two quotes, and should only be used when a symbol is impossible to represent using inline symbols (for example, the symbols \texttt{"()"} and \texttt{""""}).
    \end{minipage}

    \subsection{Expressions}\label{subsec:expressions}

    \begin{minipage}{\columnwidth}
        Expressions in M have two forms, symbols and lists.
        Symbol expressions are symbols as defined in section~\ref{subsec:symbols}.
        Lists expressions are lists of expressions surrounded by matching parentheses.
    \end{minipage}

    \subsection{Programs}\label{subsec:programs}

    \begin{minipage}{\columnwidth}
        M programs are lists of expressions.
        Note that there is no terminator specified for M programs, nor a separator for expressions, so M programs are not restricted to a single file or a null-terminated string.
    \end{minipage}
    \newpage

    \section{Semantics}\label{sec:semantics}

    \begin{figure*}[h]
        \centering
        \begin{align*}
            &\frac{\Gamma(x)=(v,m)}{\langle x,\Gamma\rangle\Downarrow\langle v,\Gamma\rangle}
            &\frac{\langle e,\Gamma\rangle\Downarrow\langle v,\Gamma\rangle}{\langle\texttt{(macro x e)},\Gamma\rangle\Downarrow\langle v,\Gamma(x) = (v,\top)\rangle}&\\
            &\frac{\Gamma}{\texttt{(fn x e)}\Downarrow\langle\lambda x.e,\Gamma\rangle}
            &\frac{\Gamma(m)=(f,\top)\quad\langle f(\Gamma,e),\Gamma\rangle\Downarrow\langle v,\Gamma\rangle}{\langle\texttt{(m e)},\Gamma\rangle\Downarrow\langle v,\Gamma\rangle}&\\
            &\frac{\langle e,\Gamma\rangle\Downarrow\langle v,\Gamma\rangle}{\langle\texttt{(def x e)},\Gamma\rangle\Downarrow\langle v,\Gamma(x) = (v,\bot)\rangle}
            &\frac{\langle f,\Gamma\rangle\Downarrow\langle\lambda x.e,\Gamma\rangle\quad\langle a,\Gamma\rangle\Downarrow\langle i,\Gamma\rangle}{\langle\texttt{(f a)},\Gamma\rangle\Downarrow\langle e[x/i],\Gamma\rangle}&
        \end{align*}
        \caption{The natural semantics of M.}
    \end{figure*}

    \subsection{Symbol}\label{subsec:symbol}

    \begin{minipage}{\columnwidth}
        Symbol expressions are expressions of the form \texttt{name}.
        They always evaluate to the \texttt{value} of the def expression that defines \texttt{name}.
        If \texttt{name} is not defined, they evaluate to $\bot$ instead.
    \end{minipage}

    \subsection{Function}\label{subsec:function}

    \begin{minipage}{\columnwidth}
        Function expressions are expressions of the form \lstinline{(fn name value)}.
        They evaluate to a function $\lambda$\texttt{name}.\texttt{value}.
        When applied, they perform the substitution \texttt{value}[\texttt{name}/\texttt{argument}].
    \end{minipage}

    \subsection{Define}\label{subsec:def}

    \begin{minipage}{\columnwidth}
        Def expressions are expressions of the form \lstinline{(def name value)}.
        They state that all symbols \texttt{name} evaluate to \texttt{value}, and evaluate to the the \texttt{value} they define.
        Multiple def expressions with the same \texttt{name} are invalid.
    \end{minipage}

    \subsection{Macro}\label{subsec:macro}

    \begin{minipage}{\columnwidth}
        Macro expressions are expressions of the form \lstinline{(macro name value)}.
        They state that all symbols \texttt{name} evaluate to \texttt{value}, and evaluate to the value of \texttt{name}.
        When applied, macros evaluate \texttt{name} and apply it to the environment and the expression of \texttt{argument}, then evaluate the result.
    \end{minipage}

    \subsection{Apply}\label{subsec:apply}

    \begin{minipage}{\columnwidth}
        Apply expressions are expressions of the form \lstinline{(value argument)}.
        If \texttt{value} is a function, it performs application as described in section~\ref{subsec:function}.
        If \texttt{value} is a macro, it performs application as described in section~\ref{subsec:macro}
    \end{minipage}
    \newpage
    \twocolumn

    \section{Encodings}\label{sec:encodings}

    \begin{minipage}{\columnwidth}
        \lstinputlisting{src/booleans.m}
    \end{minipage}
    \begin{minipage}{\columnwidth}
        \lstinputlisting{src/products.m}
    \end{minipage}
    \begin{minipage}{\columnwidth}
        \lstinputlisting{src/coproducts.m}
    \end{minipage}
    \begin{minipage}{\columnwidth}
        \lstinputlisting{src/lists.m}
    \end{minipage}
    \begin{minipage}{\columnwidth}
        \lstinputlisting{src/nats.m}
    \end{minipage}
    \begin{minipage}{\columnwidth}
        \lstinputlisting{src/chars.m}
    \end{minipage}
    \begin{minipage}{\columnwidth}
        \lstinputlisting{src/symbols.m}
    \end{minipage}
    \newpage

%    \section{Reference Implementation}\label{sec:implementation}
%
%    \subsection{Parser}\label{subsec:parser}
%
%    \lstinputlisting{src/parser.m}
%
%    \subsection{Interpreter}\label{subsec:interpreter}
%
%    \lstinputlisting{src/interpreter.m}
%
%    \subsection{Compiler}\label{subsec:compiler}
%
%    \lstinputlisting{src/compiler.m}
\end{document}
